\input{gelatin.tex}   % Import common styles.
\fancyfoot[C]{Page \thepage}
\title{\productname\ Release \productversion\\
User Documentation\\
\vspace{5 mm}
\large Converting structured text files to XML}
\author{Samuel Abels}

\begin{document}
\maketitle
\tableofcontents

\newpage
\section{Introduction}
\subsection{Why \productname?}

\product does not yet have any documentation - sorry!

\subsection{Legal Information}

\product and this handbook are distributed under the terms and conditions 
of the GNU GPL (General Public License) Version 2. You should have received 
a copy of the GPL along with \product. If you did not, you may read it here:

\vspace{1em}
\url{http://www.gnu.org/licenses/gpl-2.0.txt}
\vspace{1em}

If this license does not meet your requirements you may contact us under 
the points of contact listed in the following section. Please let us know 
why you need a different license - perhaps we may work out a solution 
that works for either of us.


\subsection{Contact Information \& Feedback}

If you spot any errors, or have ideas for improving \product or this 
documentation, your suggestions are gladly accepted.
We offer the following contact options: \\

\begin{tabular}{ll}
{\bf Bug tracker:}   & http://github.com/knipknap/Gelatin/issues \\
{\bf Jabber:}        & knipknap@jabber.org
\end{tabular}


\newpage
\section{Overview}

\product is a combined lexer, parser, and output generator. It is a simple
language for converting text into a structured formats.

\section{Example}

Suppose you want to convert the following text file to XML:

\begin{lstlisting}
User
----
Name: John, Lastname: Doe
Office: 1st Ave
Birth date: 1978-01-01

User
----
Name: Jane, Lastname: Foo
Office: 2nd Ave
Birth date: 1970-01-01
\end{lstlisting}

The following Gelatin syntax does the job:

\begin{lstlisting}
# Define commonly used data types. This is optional, but
# makes your life a litte easier by allowing to reuse regular
# expressions in the grammar.
define nl /[\r\n]/
define ws /\s+/
define fieldname /\w+/
define value /[^\r\n,]/
define field_end /[\r\n,]/

grammar user:
    match nl:
        do.return()
    match ws:
        do.next()
    match fieldname ':' ws value field_end:
        out.add('$0', '$3')

# The grammar named "input" is the entry point for the converter.
grammar input:
    match 'User' nl '----':
        out.enter('user')
        user()
\end{lstlisting}
\end{document}
